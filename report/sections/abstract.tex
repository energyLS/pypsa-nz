New Zealand's abundant resources and high share of renewable energy in the existing electricity system provide an exciting opportunity for the production and export of green hydrogen (derivatives). However, the regulation of green hydrogen in different import markets varies considerably. While the European Union imposes strict rules on additionality, temporal and geographical correlation, hydrogen regulation in countries such as Japan is less stringent. This has profound implications for New Zealand's hydrogen export potentials and prices.

To investigate these potentials and prices under different hydrogen regulations on the import side, we apply a fully sector-coupled capacity expansion and dispatch model of New Zealand across 20 regions, including integrated gas and electricity network planning based on PyPSA-Earth. The open-source model is used to simulate and optimise New Zealand's energy system in 40 scenarios at 3-hourly time resolution, sweeping through different hydrogen export volumes and hydrogen regulatory regimes.

The model results show that the prices and potentials of hydrogen exports are highly dependent on the applied hydrogen regulation. Stricter regulation, as imposed by the European Commission's Delegated Act on Union Methodology for RFNBOs, reduces grid emissions and increases the price of exported hydrogen through additional solar PV and onshore wind installations. Conversely, less stringent regulations lead to cheaper prices for exported hydrogen in countries such as Japan. The choice of import markets (and their green hydrogen regulations) plays a critical role in New Zealand's hydrogen export infrastructure and electricity supply. Accompanying policies can prevent carbon leakage while unlocking New Zealand's extraordinary potential and prerequisites.